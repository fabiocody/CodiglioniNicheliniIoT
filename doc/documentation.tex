%%%%%%%%%%%%%%%%%%%%%%%%%%%%%%%%%%%%%%%%%
\documentclass[a4paper, 11pt, parskip=half]{scrartcl}
\usepackage[english]{babel}
\usepackage{DejaVuSerif}
\usepackage{DejaVuSans}
\usepackage{sourcecodepro}
\usepackage[utf8x]{inputenc}
\usepackage[T1]{fontenc}
\usepackage{amsmath}
\usepackage{graphicx}
\usepackage[colorinlistoftodos]{todonotes}
\usepackage{hyperref}
\usepackage{float}
\usepackage{caption}

%\usepackage{fancyhdr}
%\pagestyle{fancy}

\areaset{17cm}{22.5cm}              % Set page width and height

\begin{document}
\begin{titlepage}
\pagenumbering{gobble}
\newcommand{\HRule}{\rule{\linewidth}{0.1mm}} 
\center % Center everything on the page
 

\textsc{Computer science engineering}\\[0.3cm] % heading course Number
\textsc{\Large Internet Of Things}\\[0.5cm] % heading course name
\textsc{\large Project n°1}\\[0.5cm] % Minor heading


\HRule \\[0.4cm]
{ \huge \bfseries Waste management system}\\[0.1cm] % Title of your Homework/assignment
\HRule \\[1.5cm]
 
%---------------------------------------------------------------------------------
%	AUTHOR SECTION (EDIT THE NAME and T.NO., only)
%---------------------------------------------------------------------------------

\begin{minipage}{0.4\textwidth}
\begin{flushleft} \large

Fabio \textsc{Codiglioni}\\ 919897 - 10484720  % Enter Your name and T.No.
\end{flushleft}
\begin{flushleft} \large


Alessandro \textsc{Nichelini}\\ 949880 - 10497404  % Enter Your name and T.No.
\end{flushleft}


\end{minipage}
\begin{minipage}{0.4\textwidth}
\begin{flushright} \large
\emph{Professor:} \\
Matteo \textsc{Cesana} % Supervisor's Name
\end{flushright}
\end{minipage}\\[1cm]
{\large \today}\\[1cm] % Date, change the \today to a set date if you want to be precise
\begin{figure}[H]
	\centering
	\includegraphics[width=\textwidth,height=5.7cm,keepaspectratio]{resources/polimi_logo}% \\[0.5cm] % 
\end{figure}

\vfill % Fill the rest of the page with white-space

\end{titlepage}

\tableofcontents          % Required
\listoffigures
\newpage

\pagenumbering{arabic}

\section{Abstract}
This document describes the implementation of our project for the IoT course.

We decided to implement the project number 1: \textit{Waste management system}. We used Contiki as the operating system for the IoT devices and Cooja as the simulator. We also used Node-RED to make data available to the external world and to build a light dashboard to display the system status and the transmitted messages.

We implemented all requested requirements by writing two different firmwares: one for the bins and one for the truck. We manually set up the requested node layout in Cooja and we randomly generated a second layout.

Full source code is available at our repo: \\
\url{https://github.com/fabiocody/CodiglioniNicheliniIoT}

\section{Implementation design choices}

\subsection{Architecture}

Each one of the two firmwares lives almost exclusively in a single \texttt{.c} source file (except for some utility functions in \texttt{toolkit.c}), and can be summed up by the following figures.

\begin{figure}[H]
	\centering
	\includegraphics[width=0.35\textwidth]{resources/truck_state_chart}
	\caption{FSM model of the truck.}
	\label{fig:truck-fsm}
\end{figure}

\begin{figure}[H]
    \begin{minipage}[t]{0.5\textwidth}
        \centering
        \includegraphics[width=0.95\textwidth]{resources/bin_flow_chart}
        \caption{Flowchart  of the bins.}
        \label{fig:bin-flow}
    \end{minipage}
    \hspace*{\fill}
    \begin{minipage}[t]{0.5\textwidth}
        \centering
        % TODO: Make it the truck's
        \includegraphics[width=0.95\textwidth]{resources/bin_flow_chart}
        \caption{Flowchart of the truck.}
        \label{fig:bin-flow}
    \end{minipage}
\end{figure}

\newpage
\subsection{Other implementation details}

\begin{itemize}
	\item We adopted two different transmission methods, both of them part of the native Rime communication stack:
		\begin{itemize}
			\item \href{http://contiki.sourceforge.net/docs/2.6/a01720.html}{\textbf{Broadcast}}
			\\
			The broadcast module sends packets to all local area neighbors with an a header that identifies the sender. No retransmission and acknowledgement implemented.
			\item \href{http://contiki.sourceforge.net/docs/2.6/a01738.html}{\textbf{RUnicast}} %-- \textit{Contiki's default protocol for single-hop unicast}
		    \\
		    The runicast primitive uses acknowledgements and retransmissions to ensure that the neighbor successfully receives the packet, thus we were able to avoid implementing an explicit acknowledgement system.
		\end{itemize}
	\item The random coordinates generated by each node at startup are limited between 0 (inclusive) and 50 (exclusive).
	\item The time interval between two consecutive \texttt{ALERT} messages sent out by the same node is statically set to 10 seconds.
	\item $\alpha_\mathrm{bin-bin} = 0.01$ and $\alpha_\mathrm{truck-bin} = 0.75$
\end{itemize}








\end{document}
